\input ../6001mac                               % On ALTDORF.AI.MIT.EDU  % -*- Mode:TeX -*-

\begin{document}                                

%%%%%%%%%%%%%%%%%%%%%%%%%%%%%%%%%%%%%%%%%%%%%%%%%%%%%%%%%%%%%%%%%%%%%%%%%%%%%%%

\psetheader{Fall Semester, 1991}{Problem Set 7 Solutions}       % D.~E.~Troxel

%%%%%%%%%%%%%%%%%%%%%%%%%%%%%%%%%%%%%%%%%%%%%%%%%%%%%%%%%%%%%%%%%%%%%%%%%%%%%%%

\medskip

%==============================================================================
\section{Part 1: Tutorial Problems}

No written solutions are distributed for tutorial problems. 

%==============================================================================
\section{Part 2: The Amazing Owl}

%------------------------------------------------------------------------------
\paragraph{Problem 1}

(a) The streams as given look like:

\beginlisp
(define ones     (cons-stream 1 ones))
(define integers (cons-stream 1 (add-streams ones integers)))
\null
==> (print-stream ones)
([STREAM] 1 1 1 1 1 1 1 1 1 1 1 1 1 1 1 1 1 1 1 1 1 1 1 1 1 1 \ldots
\null
==> (print-stream integers)
([STREAM] 1 2 3 4 5 6 7 8 9 10 11 12 13 14 15 16 17 18 19 20 \ldots
\endlisp

(b) We can define {\cf mystery-stream} as prescribed by making a table for
computing successive elements. We start with 1 followed by 1. Now, to compute
the next element, we add the previous element (which is, by definition, the sum
of all preceding elements) to the sum of all the elements before that. But, by
construction, the sum of all the elements before that {\it is\/} the previous
element, so we just add the previous element to itself.

\beginlisp
mystery:        1 1 2 4 \ldots 2$^{i-2}$ \ldots
sum previous:   0 1 2 4 \ldots 2$^{i-2}$ \ldots
             + -------------------
next:           1 2 4 8 \ldots 2 $\cdot$ 2$^{i-2}$ = 2$^{i-2+1}$ \ldots
\endlisp

Similar reasoning leads to the realization that the third and following
elements are all computed by just adding the {\cf mystery-stream} to itself
starting from the second element. In short:

\beginlisp
(define mystery-stream
 (cons-stream 1
              (cons-stream 1
                           (add-streams (tail mystery-stream)
                                        (tail mystery-stream)))))
\null
==> (print-stream mystery-stream)
([STREAM] 1 1 2 4 8 16 32 64 128 256 512 1024 2048 4096 8192 16384 \ldots
\endlisp

As the above tabulation suggests, the tail of {\cf mystery-stream} is just the
stream of successive powers of 2, starting with $2^0 = 1$.

If we were allowed to use an auxilary generating procedure (which the problem
statement forbade), the following would have been equivalent:

\beginlisp
(define mystery-stream (cons-stream 1 (sum-stream mystery-stream)))
\null
(define (sum-stream st)
  (define sum (cons-stream (head st)                     ; ``previous'' elt = sum prev
                           (add-streams sum (tail st)))) ; always add prev to current
  sum)
\endlisp

or, more directly,

\beginlisp
(define (mystery-maker n)
  (cons-stream n (mystery-maker (+ n n))))   ; or (* 2 n) since n+n = 2n
\null
(define mystery-stream (cons-stream 1 (mystery-maker 1)))
\endlisp

(c) We use {\cf random}, which was also used in problem set~2. Since {\cf
random} accepts only an upper bound, it is wasteful to do the obvious hack of
generating an integer less than our given upper bound then discard and try
again if this happens to be less than our lower bound. Instead, we consider the
width of the interval between L and U and generate random numbers within this
width. By adding these randomly generated width numbers to the lower bound, we
always produce a random integer within the desired interval. The code:

\beginlisp
(define (noise-stream lower-bound upper-bound)
  (let ((difference (- upper-bound lower-bound)))
    (cons-stream (+ lower-bound (random difference))
                 (noise-stream lower-bound upper-bound))))
\null
==> (print-stream (noise-stream 3 8))
([STREAM] 3 6 3 4 4 4 7 4 3 4 7 6 4 5 6 6 4 3 7 7 3 5 4 7 5 \ldots
\endlisp

Notice that, as desired, our lower bound appears in the noise stream, but our
upper bound never does.

%------------------------------------------------------------------------------
\paragraph{Problem 2}

The most obvious solution is to define a new procedure {\cf sub-streams} to
subtract two streams, but we can avoid this by simply scaling the subtracted
stream by $-1$ then using {\cf add-streams}.  The code which subtracts from
each element its predecessor in the stream (with the conventional 0 starting
entry) is:

\beginlisp
(define (diff st)
  (add-streams (tail st)
               (scale-stream -1 st)))
\null
==> (print-stream (diff integers))
([STREAM] 1 1 1 1 1 1 1 1 1 1 1 1 1 1 1 1 1 1 1 1 1 1 1 1 1 1 1 1 \ldots
\null
==> (print-stream (diff mystery-stream))
([STREAM] 0 1 2 4 8 16 32 64 128 256 512 1024 2048 4096 8192 16384 \ldots
\endlisp

As expected, {\cf (diff mystery-stream)} yields the successive powers of 2.

\newpage

%------------------------------------------------------------------------------
\paragraph{Problem 3}

Averaging of successive values can be done by adding the successive values than
dividing by 2, that is, scaling by $1/2$.

\beginlisp
(define (smooth stream)
  (scale-stream 0.5 (add-streams stream (tail stream))))
\null
==> (print-stream (smooth integers))
([STREAM] 1.5 2.5 3.5 4.5 5.5 6.5 7.5 8.5 9.5 10.5 11.5 12.5 13.5 \ldots
\endlisp

%------------------------------------------------------------------------------
\paragraph{Problem 4}

\mbox{} % new paragraph

(a) One nice abstract way to achieve a repeated smoothing is to use the
definition of {\cf repeated} from problem set~2 and from Quiz~1.

\beginlisp
(define (repeated f n)
  (if (= n 0)
      (lambda (z) z)
      (compose f (repeated f (- n 1)))))
\null
(define (compose f g)
  (lambda (x) (f (g x))))
\null
(define (smooth-n stream n)
  ((repeated smooth n) stream))
\null
==> (print-stream (smooth-n integers 0))
([STREAM] 1 2 3 4 5 6 7 8 9 10 11 12 13 14 15 16 17 18 19 20 \ldots
==> (print-stream (smooth-n integers 1))
([STREAM] 1.5 2.5 3.5 4.5 5.5 6.5 7.5 8.5 9.5 10.5 11.5 12.5 13.5 \ldots
==> (print-stream (smooth-n integers 2))
([STREAM] 2 3 4 5 6 7 8 9 10 11 12 13 14 15 16 17 18 19 20 \ldots
==> (print-stream (smooth-n integers 3))
([STREAM] 2.5 3.5 4.5 5.5 6.5 7.5 8.5 9.5 10.5 11.5 12.5 13.5 \ldots
\endlisp

Amusing hypothesis: (1) for even numbers of smoothing, say $2k$ we get the
integers starting at $k+1$.  (2) for odd numbers of smoothing, say $2k+1$,
we get the sum of the stream of 0.5 and the integers starting at $k+1$

\beginlisp
==> (print-stream (smooth-n integers 42))  ; 2*21
([STREAM] 22 23 24 25 26 27 28 29 30 31 32 33 34 35 36 37 38 39 40 \ldots
==> (print-stream (smooth-n integers 37))   ; 2*18 + 1
([STREAM] 19.5 20.5 21.5 22.5 23.5 24.5 25.5 26.5 27.5 28.5 29.5 30.5 \ldots
\endlisp

\newpage

(b) We chose to generate a signal with just a bit of noise.

\beginlisp
(define        4b-sig (car (make-test-signal 3)))
(define smooth-4b-sig (smooth-n 4b-sig 4))
(define   diff-4b-sig (diff     4b-sig))
\null
==> (print-stream 4b-sig)
([STREAM] 0 2 0 1 2 0 1 0 1 2 1 0 2 2 2 2 2 0 1 1 1 0 2 2 0 2 0 2 0 0 2 0 1 2 2
          1 0 2 2 1 2 2 2 0 1 1 1 1 1 0 2 2 0 0 2 1 1 0 1 2 0 2 1 0 2 1 1 2 0 1
          0 0 0 1 2 1 2 0 0 1 2 1 4987.4673 4987.4673 4988.4673 4987.4673 \ldots
\null
==> (print-stream smooth-4b-sig)
([STREAM] 0.875  1.0    1.0625 0.8125 0.5625 0.625  1.0    1.25   1.0625 1.0
          1.4375 1.875  2.0    1.875  1.4375 0.9375 0.8125 0.875  0.8125 0.9375
          1.3125 1.375  1.125  1.0    1.0    0.875  0.625  0.625  0.8125 0.875
          1.125  1.5625 1.5625 1.125  1.0    1.375  1.625  1.625  1.75   1.8125
          1.4375 0.9375 0.8125 0.9375 1.0    0.9375 0.8125 0.9375 1.3125 1.25
          0.75   0.6875 1.0625 1.125  0.8125 0.6875 0.9375 1.125  1.125  1.125
          1.0    0.9375 1.125  1.25   1.25   1.125  0.8125 0.5    0.25   0.125
          0.375  0.9375 1.375  1.4375 1.125  0.625  0.5    0.9375 312.9667
          1559.521 3429.3213 4676.0635 4987.8423 4987.78 4987.905 \ldots
\null
==> (print-stream diff-4b-sig)
([STREAM]  2 -2  1  1 -2  1 -1  1  1 -1 -1  2  0  0  0  0 -2  1  0  0 -1  2  0
          -2  2 -2  2 -2  0  2 -2  1  1  0 -1 -1  2  0 -1  1  0  0 -2  1  0  0
           0  0 -1  2  0 -2  0  2 -1  0 -1  1  1 -2  2 -1 -1  2 -1  0  1 -2  1
          -1  0  0  1  1 -1  1 -2  0  1  1 -1 4986.4673 0  1 -1 \ldots
\null
==> (print-stream (diff smooth-4b-sig))
([STREAM] 0.125   0.0625 -0.25   -0.25    0.0625  0.375   0.25   -0.1875 -0.0625
          0.4375  0.4375  0.125  -0.125  -0.4375 -0.5    -0.125   0.0625 -0.0625
          0.125   0.375   0.0625 -0.25   -0.125   0.0    -0.125  -0.25    0.0
          0.1875  0.0625  0.25    0.4375  0.0    -0.4375 -0.125   0.375   0.25
          0.0     0.125   0.0625 -0.375  -0.5    -0.125   0.125   0.0625 -0.0625
         -0.125   0.125   0.375  -0.0625 -0.5    -0.0625  0.375   0.0625 -0.3125
         -0.125   0.25    0.1875  0.0     0.0    -0.125  -0.0625  0.1875  0.125
          0.0    -0.125  -0.3125 -0.3125 -0.25   -0.125   0.25    0.5625  0.4375
          0.0625 -0.3125 -0.5    -0.125   0.4375 312.0292 1246.5543 1869.8003
          1246.7422  311.7788    -0.0625  0.125   0.4375  0.25 0.0 \ldots
\null
==> (print-stream (smooth-n diff-4b-sig 4))
([STREAM] 0.125   0.0625 -0.25   -0.25    0.0625  0.375   0.25   -0.1875 -0.0625
          0.4375  0.4375  0.125  -0.125  -0.4375 -0.5    -0.125   0.0625 -0.0625
          0.125   0.375   0.0625 -0.25   -0.125   0.0    -0.125  -0.25    0.0
          0.1875  0.0625  0.25    0.4375  0.0    -0.4375 -0.125   0.375   0.25
          0.0     0.125   0.0625 -0.375  -0.5    -0.125   0.125   0.0625 -0.0625
         -0.125   0.125   0.375  -0.0625 -0.5    -0.0625  0.375   0.0625 -0.3125
         -0.125   0.25    0.1875  0.0     0.0    -0.125  -0.0625  0.1875  0.125
          0.0    -0.125  -0.3125 -0.3125 -0.25   -0.125   0.25    0.5625  0.4375
          0.0625 -0.3125 -0.5    -0.125   0.4375 312.0292 1246.5543 1869.8003
          1246.7422  311.7788    -0.0625  0.125   0.4375  0.25 0.0 \ldots
\endlisp

The order of application of {\cf smooth} and {\cf diff} apparently doesn't
matter.

\newpage

%------------------------------------------------------------------------------
\paragraph{Problem 5}

A straigthforward use of {\cf map} will do the trick\ldots

\beginlisp
(define (mark smooth-diff-stream mean dev)
  (map (lambda (elt) (if (> (abs (- elt mean)) (* 3 dev))
                         1
                         0))
       smooth-diff-stream))
\null
(define smooth-diff (smooth-n diff-4b-sig 4))
\null
==> (print-stream (mark smooth-diff mean dev))
([STREAM] 0 0 0 0 0 0 0 0 0 0 0 0 0 0 0 0 0 0 0 0 0 0 0 0 0 0 0 0 0 0 0 0 0
          0 0 0 0 0 0 0 0 0 0 0 0 0 0 0 0 0 0 0 0 0 0 0 0 0 0 0 0 0 0 0 0 0
          0 0 0 0 0 0 0 0 1 1 1 1 1 0 0 0 0 0 0 0 0 0 0 0 0 0 0 0 0 0 0 0 0
          0 0 0 0 0 0 0 0 0 0 0 0 0 0 0 0 0 0 0 0 0 0 0 0 0 0 0 0 0 0 0 \ldots
\endlisp

The 1's are 75 elements into the stream and there are 5 of them.

%------------------------------------------------------------------------------
\paragraph{Problem 6}

We can exploit our definition of {\cf smooth} by just multiplying the smoothed
{\cf smooth-st} by the {\cf marked-st}, since multiplying by 0 will yield the
desired 0 while multiplying the smoothed {\cf smooth-st} by 1 yields the
desired averaged elt and successor.

\beginlisp
(define (combine-streams proc s1 s2)
  (cond ((empty-stream? s1) s2)
        ((empty-stream? s2) s1)
        (else
         (cons-stream (proc (head s1) (head s2))
                      (combine-streams proc (tail s1) (tail s2))))))
\null
(define (mult-streams st1 st2)
  (combine-streams  * st1 st2))
\null
(define (get-signal-strength marked-st smooth-st)
  (mult-streams marked-st (smooth smooth-st)))
\null
==> (print-stream (mult-streams integers integers))   ; perfect squares
([STREAM] 1 4 9 16 25 36 49 64 81 100 121 144 169 196 225 \ldots
\null
==> (define punt6 (get-signal-strength (mark integers 0 2)     ; punt first 6
                                       (smooth-n integers 3))) ; 2.5...
==> (print-stream punt6)
([STREAM] 0 0 0 0 0 0 9 10 11 12 13 14 15 \ldots
==> (define punt9 (get-signal-strength (mark integers 0 3)     ; punt first 9
                                       (smooth-n integers 3))) ; 2.5...
==> (print-stream punt9)
([STREAM] 0 0 0 0 0 0 0 0 0 12 13 14 15 16 \ldots
\endlisp

Good.  We expected to punt the first 6 elts of {\cf punt6} since our mean was 0
and deviation was 2 (so we look for $2\sigma = 6$) then returned the once-more
smoothed (i.e., averaged smoothed value and its successor)
thrice-smoothed integers, which we know will be the integers starting
at 3 (i.e., $4 = 2 \times 2$ so start at $2+1$).  Thus, we punt 3--8 (first 6)
and resume integers at 9. Similary, {\cf punt9} will punt the first 9 of the
same stream.

%------------------------------------------------------------------------------
\paragraph{Problem 7}

We use {\cf filter} to pass only those entries which are not 0 (notice that
negative values {\it can\/} occur if our smoothed stream is decreasing in
values. Neither {\cf integers} not {\cf mystery-stream} do, but in general a
stream might.  We chose to group times and values by making a {\cf cons} pair.

\beginlisp
(define (dev-posns str)
  (filter (lambda (x) (not (= (car x) 0)))
          (combine-streams cons str integers)))
\endlisp

Since the stream produced by {\cf dev-posns} is a stream of compound data---
specifically, one in which each element is a pair--- it would also be
nice to define selectors to go along with the constructor.

\beginlisp
(define (posn-value devposnstr) (car devposnstr))
(define (posn-time  devposnstr) (cdr devposnstr))
\endlisp

\beginlisp
==> (print-stream (dev-posns punt6))
([STREAM] (9 . 7) (10 . 8) (11 . 9) (12 . 10) (13 . 11) (14 . 12) \ldots
\endlisp

As expected.

%------------------------------------------------------------------------------
\paragraph{Problem 8}

As you can see, our above abstractions paid off\ldots

\beginlisp
(define (measure-time-spread-and-signals l-str r-str)
  (combine-streams (lambda (event1 event2)
                     (list (- (posn-time event1)
                              (posn-time event2))
                           (posn-value event1)
                           (posn-value event2)))
                   l-str
                   r-str))
\endlisp

Since the stream produced by this procedure is also a stream of compound
data--- spec., one in which each element is a triple--- it would also be
nice to define selectors to go along with this constructor too.

\beginlisp
(define  time-shift car)
(define  left-value cadr)
(define right-value caddr)
\endlisp

We can test this by checking the time spread of the deviation positions of {\cf
punt6} and {\cf punt9}. We should see time deviations of -3 since {\cf punt6}
anticipates {\cf punt9} by three time stamps. Also, the corresponding values
should always be 3 higher for {\cf punt9} since it is essentially a three unit
delayed version of {\cf punt6}. Let's see\ldots

\beginlisp
==> (print-stream (measure-time-spread-and-signals (dev-posns punt6) (dev-posns punt9)))
([STREAM] (-3 9 12) (-3 10 13) (-3 11 14) (-3 12 15) (-3 13 16) \ldots
\endlisp

Roger dodger.

%------------------------------------------------------------------------------
\paragraph{Problem 9}

First, we define our mean and deviation to be the same as was provided in the
problem set code listing, just to be on the safe side.

\beginlisp
(define mean 0) ;the derivative of a bounded signal has zero average value
(define dev 20) ;who knows what evil lurks....
\endlisp

Next, let's define something to process the signals by smoothing
them, marking the differentiations, and finally hacking out the signal strength
and time-stamping the deviation positions.

\beginlisp
(define (process-signal str smooth-factor)
  (let ((smoothed-str (smooth-n str smooth-factor)))
    (dev-posns (get-signal-strength (mark (diff smoothed-str) mean dev)
                                    smoothed-str))))
\endlisp

Now we can do the simulation by applying {\cf compute-distance-and-angle} to
each element (which is a triple!) of the stream produced by {\cf
measure-time-spread-and-signals}.

\beginlisp
(define (simulate l-str r-str smooth-factor v b)
  (map (lambda (spread-n-sigs) (compute-distance-and-angle ( time-shift spread-n-sigs)
                                                           ( left-value spread-n-sigs)
                                                           (right-value spread-n-sigs)
                                                           v b))
       (measure-time-spread-and-signals (process-signal l-str smooth-factor)
                                        (process-signal r-str smooth-factor))))
\endlisp

Now lets try it out on (make-test-signal 2)

\beginlisp
(define       signals (make-test-signal 2))
(define  left-signal  (car signals))
(define right-signal  (cdr signals))
\null
==> (head (simulate left-signal right-signal 2 signal-velocity baseline))
(70.13121 0.77539754)
\endlisp

Looks like mickey has a distance of about 70 (in unknown units).
He is oriented at an angle of 0.775 radians or about 44.4 degrees.


%------------------------------------------------------------------------------
\paragraph{Problem 10}

First, a recursive version (employs wishful thinking)

\beginlisp
(define (first-n str n)
  (if (= n 0)
      (cons 0 str)
      (let ((pair (first-n (tail str) (-1+ n))))
        (cons (+ (car pair) (head str))
              (cdr pair)))))
\endlisp

And now an iterative version

\beginlisp
(define (first-n str n)
  (define (iter str n acc)
    (if (= n 0)
        (cons acc str)
        (iter (tail str) (-1+ n) (+ acc (head str)))))
  (iter str n 0))
\endlisp

Now for a handy test\ldots

\beginlisp
==> (let ((first-100 (first-n integers 100)))
      (cons (car first-100) (head (cdr first-100))))
(5050 . 101)
\endlisp

\newpage

%------------------------------------------------------------------------------
\paragraph{Problem 11}

(a) Here is a literal reading of the suggested approach.

\beginlisp
(define (set-up-sums st n)
  (let ((start (first-n st n)))
    (let ((sum   (car start))
          (front (cdr start))
          (back st))
     (define running-sum
       (cons-stream sum
                    (add-streams running-sum
                                 (add-streams front
                                              (scale-stream -1 back)))))
     running-sum)))
\endlisp

(b) We can now reduce {\cf set-up-squares} to a call to {\cf set-up-sums}.

\beginlisp
(define (set-up-squares st n)
  (set-up-sums (map square st) n))
\endlisp

(c) Testing on our trusty integers\ldots

\beginlisp
==> (print-stream (set-up-sums integers 3))  ; should be multiples of 3:
                                             ;  (n-1) + n + (n+1) = 3n
([STREAM] 6 9 12 15 18 21 24 27 30 33 36 39 42 45 48 51 54 57 \ldots
\null
==> (print-stream (set-up-squares integers 2))
([STREAM] 5 13 25 41 61 85 113 145 181 225 \ldots
\endlisp

%------------------------------------------------------------------------------
\paragraph{Problem 12}

The code

\beginlisp
(define (mean-st st n)
  (map (lambda (x) (/ x n))
       (set-up-sums st n)))
\endlisp

The test

\beginlisp
==> (print-stream (mean-st integers 3))  ; should coerce the 3n into n's
([STREAM] 2 3 4 5 6 7 8 9 10 11 12 13 14 15 16 17 18 19 20 22 23 \ldots
\endlisp

Not surprisingly, this is the average of the first three (which is the second),
the next three (the third) and so on. Integers are just nicely balanced that
way.

\newpage

%------------------------------------------------------------------------------
\paragraph{Problem 13}

\mbox{} % end paragraph

\beginlisp
(define (dev-st st n)
  (map (lambda (x) (sqrt (abs x)))
       (add-streams (map (lambda (x) (/ x n))
                         (set-up-squares st n))
                    (scale-stream -1 (map square (mean-st st n))))))
\endlisp

For $n$ = 1, we subtract a square from itself yielding 0, whose sqrt is 0.

For $n$ = 2, we average a number's square and the square of its increment then
subtract their average squared. This is:

\begin{eqnarray*}
\sqrt{1/2(n^2 + (n+1)^2) - (1/2(n+n+1))^2} &=& \sqrt{n^2+n+1/2 - (n+1/2)^2} \\
                                           &=& \sqrt{1/2 - 1/4}             \\
                                           &=& \sqrt{1/4}                   \\
                                           &=& 1/2
\end{eqnarray*}

For $n$ = 3, we average a numbers square with the square of its two closest neighbors
then subtract their average (which is the center number) squared. This is:

\begin{eqnarray*}
\sqrt{1/3((n-1)^2 + n^2 + (n+1)^2) - (1/3((n-1)+n+(n+1)))^2} &=& \sqrt{(n^2+2/3 - n^2} \\
                                                             &=& \sqrt{2/3}            \\
                                                             &=& 0.8164966
\end{eqnarray*}

\beginlisp
==> (print-stream (dev-st integers 1))
([STREAM] 0 0 0 0 0 0 0 0 0 0 0 0 0 0 0 0 0 0 0 0 0 0 0 \ldots
\null
==> (print-stream (dev-st integers 2))
([STREAM] 0.5 0.5 0.5 0.5 0.5 0.5 0.5 0.5 0.5 0.5 0.5 0.5 \ldots
\null
==> (print-stream (dev-st integers 3))
([STREAM] 0.8164966 0.8164966 0.8164966 0.8164966 0.8164966 \ldots
\endlisp

The proof is in the pudding.

%------------------------------------------------------------------------------
\paragraph{Problem 14}

To produce a list of three elements, we cons an element onto a two-element
list.

\beginlisp
(define (glued-data st n)
  (combine-streams CONS                              ;; **
                   (nth-tail (round (/ n 2)) st)
                   (combine-streams LIST             ;; **
                                    (mean-st st n)
                                    ( dev-st st n))))
\endlisp

Better define selectors when we define a constructor (else Captain Abstraction
might reap his vengeance)!

\beginlisp
(define glue-val  car)
(define glue-mean cadr)
(define glue-dev  caddr)
\endlisp

\beginlisp
==> (print-stream (glued-data integers 1))
([STREAM] (1 1 0) (2 2 0) (3 3 0) (4 4 0) (5 5 0) \ldots
\null
==> (print-stream (glued-data integers 2))
([STREAM] (2 3/2 0.5) (3 5/2 0.5) (4 7/2 0.5) (5 9/2 0.5) (6 11/2 0.5) \ldots
\null
==> (print-stream (glued-data integers 3))
(3 2 0.8164966) (4 3 0.8164966) (5 4 0.8164966) (6 5 0.8164966) (7 6 0.8164966) \ldots
\endlisp

No surprises.

%------------------------------------------------------------------------------
\paragraph{Problem 15}

\mbox{} % end paragraph

\beginlisp
(define (new-mark glued-str)
  (map (lambda (elt)
         (let ((val  (glue-val  elt))
               (mean (glue-mean elt))
               (dev  (glue-dev  elt)))
           (if (> (abs (- val mean)) (* 3 dev))
               1
               0)))
       glued-str))
\endlisp

\beginlisp
==> (print-stream (new-mark (glued-data smooth-diff 45)))
([STREAM] 0 0 0 0 0 0 0 0 0 0 0 0 0 0 0 0 0 0 0 0 0 0 0 0 0 0 0 0 0 0 0 0 0
          0 0 0 0 0 0 0 0 0 0 0 0 0 0 0 0 0 0 1 1 1 0 0 0 0 0 0 0 0 0 0 0 0
          0 0 0 0 0 0 0 0 0 0 0 0 0 0 0 0 0 0 0 0 0 0 0 0 0 0 0 0 0 \ldots
\endlisp

Here the 1's are 52 elements into the stream and there are only 3 of them (the
first and last of the original 5 were smoothed out by taking the mean and
deviation over an $n$-element wide range).

%------------------------------------------------------------------------------
\paragraph{Problem 16}

Now for the new version of our simulator which is remarkably
similar to the earlier version!

(a) As before, let's define something to process the signals by smoothing them,
marking the differentiations, and finally hacking out the signal strength and
time-stamping the deviation positions.

This time, though, we need to add {\cf run-length} as a new parameter since
{\cf new-mark} processes the glued string and {\cf glued-str} needs the {\cf
run-length} to calculate the mean and standard deviation by a running average.

\beginlisp
(define (new-process-signal str smooth-factor run-length)
  (let ((smoothed-str (smooth-n str smooth-factor)))
    (dev-posns (get-signal-strength (new-mark (glued-data (diff smoothed-str) run-length))
                                    (nth-tail (round (/ run-length 2)) smoothed-str)))))
\endlisp

Now, as before, we can do the simulation by applying {\cf
compute-distance-and-angle} to each element (which is a triple!) of the stream
produced by {\cf measure-time-spread-and-signals}.

\beginlisp
(define (new-simulate l-str r-str smooth-factor v b run-length)
  (map (lambda (spread-n-sigs) (compute-distance-and-angle ( time-shift spread-n-sigs)
                                                           ( left-value spread-n-sigs)
                                                           (right-value spread-n-sigs)
                                                           v b))
       (measure-time-spread-and-signals (new-process-signal l-str smooth-factor run-length)
                                        (new-process-signal r-str smooth-factor run-length))))
\endlisp

(b) Now lets try it out

\beginlisp
==> (head (new-simulate left-signal right-signal 2 signal-velocity baseline 45))
(70.05803 0.77539754)
\endlisp

Looks like mickey still has a distance of about 70 (in unknown units).
He is still oriented at an angle of 0.775 radians or about 44.4 degrees.

(c) Now for some multi-signal testing

\beginlisp
(define last-signals      (make-test-signal-multi 3))
(define last-left-signal  (car last-signals))
(define last-right-signal (cdr last-signals))
\null
(define final-simulation
  (new-simulate last-left-signal last-right-signal 2 signal-velocity baseline 55))
\null
==> (print-stream final-simulation)
([STREAM] (70.04085661064 .7753974966108)   ;; 45 degrees
          (69.99764554181 .7753974966108)   ;; 45 degrees
          (50.52200199431 .7753974966108)   ;; 45 degrees
          (50.5014191151  .7753974966108)   ;; 45 degrees
          (20.96721132096 .7075844367254)   ;; 40 degrees
          (20.96669612222 .7075844367254)   ;; 40 degrees
          \vdots
          <infinite-loop>
\endlisp

These are the only significant events since {\cf make-test-signal-multi} passes
{\cf d-theta-onset-values} as the {\cf d-theta-onset-list} argument to {\cf
make-audio-signal-multi} and {\cf d-theta-onset-values} contains only three events. We
see both the onset and offset of each event, so we detect a total of six
interesting movements.

Mickey seems to be getting closer and circling toward our right flank. He's
walking softly into that sweet night.

(d) In this simulation the value {\cf signal-velocity} was 1 and {\cf baseline}
was 10. Moreover, the time difference between the ears was forced to be an
integer. This combination of facts means that the time difference between the
two ears was always either -2, -1, 0, 1, or 2. Thus there are only five possible
perceived angles for the prey. With this low accuracy our simulated owl would
find itself on a very strick diet\ldots ultra-ultra slim fast!

The accuracy could be increased by making the velocity small relative to the
baseline length. this is analogous to a higher rate of sampling for the ear
signals.

%==============================================================================

\end{document}
