\input /sw/6001/6001mac				% On Altdorf.AI.MIT.EDU

\def\fbox#1{%
  \vtop{\vbox{\hrule%
     \hbox{\vrule\kern3pt%
 \vtop{\vbox{\kern3pt#1}\kern3pt}%
 \kern3pt\vrule}}% 
 \hrule}}

\begin{document}

%%%%%%%%%%%%%%%%%%%%%%%%%%%%%%%%%%%%%%%%%%%%%%%%%%%%%%%%%%%%%%%%%%%%%%%%%%%%%%%

\psetheader{Fall Semester, 1991}{Problem Set 2}

%%%%%%%%%%%%%%%%%%%%%%%%%%%%%%%%%%%%%%%%%%%%%%%%%%%%%%%%%%%%%%%%%%%%%%%%%%%%%%%

\noindent
Issued: Tuesday, September 17

\noindent
Tutorial preparation for:  Week of September 23

\noindent
Written solutions due: Friday, September 27, in Recitation


\noindent
Reading assignment: Chapter 1, Sections 1.2 and 1.3



\vskip 20pt

As noted in Problem Set 1, every homework assignment describes two
sorts of activities:

\begin{itemize}

\item {\it Tutorial preparation:}  There are some questions that you
should be ready to present orally in tutorial.  Your tutor may choose
not to cover every question every week, but you should be prepared to
discuss them.  You should not write up formal answers to these, other
than making notes for yourself, if you choose.

\item {\it Written assignments:}  These are due in recitation the end
of the following week.  Solutions should always be handed in at
recitation, and {\bf late work will not be accepted.}  You can begin working
on the assignment now, and through next week.  It is to your advantage
to get lab work done early, rather than waiting until the night before
it is due.  Your tutor will look over the homework you handed in, and
review it with you in tutorial.

\end{itemize}

%==========================================================================
\section{Part 1.  Tutorial exercises}

You should prepare these exercises for oral presentation in tutorial.

\begin{itemize}

%----------------------------------------------------------------------
\item Exercise 1. -- Do exercise 1.12  in the textbook.

%----------------------------------------------------------------------
\item Exercise 2. -- Do exercise 1.13  in the textbook.  The solution
to exercise 1.11 is given below, to help you in doing this exercise.

\beginlisp
(define (new-exp b n)
  (define (square x) (* x x))
  (define (iter-exp multiplier product exponent)
    (cond ((= exponent 0) product)
          ((even? exponent)
           (iter-exp (square multiplier) product (/ exponent 2)))
          (else (iter-exp multiplier
                          (* product multiplier)
                          (- exponent 1)))))
  (iter-exp b 1 n))
\endlisp

%----------------------------------------------------------------------
\item Exercise 3. -- Assume that we have evaluated the following
expressions:

\beginlisp
(define foo (lambda (m) (lambda (z) (m z))))
\null
(define square (lambda (x) (* x x)))
\endlisp

Using the substitution model, show
how the following expressions evolve when being evaluated.

(a)  \beginlisp((lambda (x) ((foo x) square)) 3)\endlisp

(b)  \beginlisp(define bar ((lambda (g) (lambda (y) (g y))) foo))\endlisp

(c)  \beginlisp((bar foo) square)\endlisp

%----------------------------------------------------------------------
\item Exercise 4. -- Consider the following higher order procedure
      {\tt repeated}, which takes a function {\tt f} of one argument, and
      a number {\tt n}, and returns a procedure which, when applied to an argument
      {\tt a}, computes 
${ \underbrace{\hbox{\tt (f (f (f \ldots (f}}_{\hbox{\rm n times}}
\hbox{\tt a))))}}$
      The procedure is given below:

\beginlisp
      (define repeated
        (lambda (f n)
          (if (= n 0)
              (lambda (z) z)
              (compose f (repeated f (- n 1))))))
\null
      (define compose
        (lambda (f g)
          (lambda (x) (f (g x)))))
\endlisp

      Assume that we now define a function called {\tt add-5} as follows:

\beginlisp
      (define add-5 (lambda (y) (+ y 5)))
\endlisp

      Using your intuition, predict and describe the object returned
      as a result of evaluating the expression

\beginlisp
      (repeated add-5 2)
\endlisp

[Note: It may be painful to work this through carefully using the
substitution model (one of the reasons why we want the computer to do
this sort of work for us).  We suggest that in using the substitution
model to trace this example, you try to skip the details and
concentrate on the returned object at each stage of the recursion,
putting these returned objects together carefully.]

\end{itemize}

%==========================================================================
\section{Part 2. Laboratory Assignment: RSA Cryptography System}

\fbox{\hbox{\vbox{This laboratory assignment, like the ones to follow,
involves a series of programming tasks.   As a matter of general
principle, whenever you are asked to write or complete some
implementation, you should not only turn in a listing of your code,
you should {\bf always} turn in a printout of an interactive {\sc Scheme}
session as an example of testing your code.}}}


Cryptography systems are used to encode and decode information for
secure transmission and storage.  They are typically based on the
notion of using ``keys'' for encryption and decryption.  A key is a method
for converting characters of the input message into a new set of
encoded characters for actual transmission.  If the receiver of the
message knows the key used to encrypt it, he can then use that key to
decrypt the received message back into the original text.  Keys can be
as simple as a table that pairs input characters with output
characters, although real encryption systems are much more complex.

The RSA Cryptography System is based on the idea of using very large
prime numbers to determine the keys for encryption and decryption of
messages.  The algorithms that we will be using for testing, and thus
finding, large prime numbers efficiently are based on Section 1.2.6 of
the text.  You should read that material before beginning work on this
assignment.

The RSA scheme belongs to the family of Public Key Cryptography Systems which
are used by corporations, the government or even individuals to send secret
messages to each other via public wires.  A cryptography system must ensure
the privacy of such communication as well as prevent forgeries.
The RSA scheme is the first and probably still the best way of realizing such
communications.  It was invented by Rivest, Shamir and Adleman (all at the
time at MIT, in course 6-3, 18 and 18 respectively.)  It is based on the
following widely accepted hypothesis:

\begin{quote}
If $n=pq$, where $p$ and $q$ are large prime numbers (say 100 digits
long), then it is in principle and in practice essentially impossible
in any reasonable length of time, given the number $n$, to find its
two prime factors $p$ and $q$.\footnote{Quoted from Professor Arthur
Mattuck's 18.063 notes.} 
\end{quote}

In the RSA scheme, a user selects two secret 100 digit primes, $p$ and $q$.
She then defines
\begin{eqnarray*}
n&=& pq\\
m&=&(p - 1)(q - 1).
\end{eqnarray*}

She also selects a 100-digit (or larger) encryption number $e$, such
that $e$ and $m$ are co-primes, i.e., the only common factor of $e$
and $m$ is 1.  (Remember that any integer can be represented uniquely
by its factors, which are those integers that divide the original one
without remainder.)  The user puts $n$ and $e$ on public file, but
keeps $m$ secret.  To encrypt a message, $s$, which is represented by
a number,\footnote{Of course, if the message is some textual string, we
first convert it to a number, e.g.~by using the position in the
alphabet to convert each letter to a number, plus some other numbers
for other characters.} we perform the following calculation:

\smallskip
\centerline{encrypted message = $s$ to the power of $e$, modulo $n$}
or 
\[S = (s^e) \bmod n.\]

Remember that modular arithmetic means that we keep only the remainder
of a number relative to a modular base.  For example, the number 100
modulo 8 is given by 4, since that is the remainder that is left after
dividing 100 by 8.

Given an encrypted message, the receiver wants to decrypt it to obtain
the original message.  In order to do this, she wants to
perform the following calculation, with some number $d$:

\smallskip
\centerline{$s^\prime$ = encrypted message to the power of $d$, modulo
$n$}
or
\[s^\prime = (S^d) \bmod n.\]

In order for this to work, the receiver needs to find a number $d$ such
that $s = s^\prime$.  You will notice that the requirement for the
number $d$ is that:   

\centerline{$s$ = $s$ to the power of $e$, then to the power of $d$, modulo
$n$.}

The equation for finding the number $d$ is:

\centerline{$d = (1 - mr)/e$, where $r$ is a negative integer, and $d$ a positive one.}

(Since this course is not 18.063, we won't ask you to understand all of the
above --- you'll have to do that when you take that course.)  In
general, this means that if the receiver knows the sender's value for
$m$, she can simply search for values of $r$ until she gets a value
for $d$, by the above definition, that is an integer.  On the other
hand, if someone else intercepts the message, to decrypt it they must
first try to find $m$, and then find $r$ (and hence $d$).  Finding $m$ is
in principle very hard, since it requires factoring $n$ into its two
prime components, and if $p$ and $q$ are large primes, this is
generally hard to do.

So much for theory: let's see some practical use of RSA.  Before we take
off, you need to set up a file in which to put your solutions.  All the
procedures in the appendix of the problem set are installed on the
Chipmunks, and can be loaded onto your floppy disk.  To do this,
follow the instructions given in the Chipmunk manual in the section
``Loading Problem Set Files'' to load the code for problem set 2.
You can now use this code, modify it and add your solutions to it.
The {\tt fast-prime?} procedure and all the sub-procedures that it uses
are taken from Section 1.2.6 of the textbook. You are also provided with a
public key selection procedure called {\tt select-keys}, which uses
procedures
{\tt select-prime, select-e} and {\tt gcd} (greatest common divisor). 
Take a look at
these procedures and make sure you understand them before trying them out.

%==========================================================================
\section{The Story}

Our friend Louis Reasoner has just joined the CIA upon receiving his
B.S.\ degree from WIT (Worthwhile Institute of Technology).  As a result of
his specialization in cryptography at WIT, he was assigned to implement a
new RSA system for the CIA, who are planning to use it for negotiating a
highly secret business deal with Iraq.

%----------------------------------------------------------------------
\section{Exercise 1}

The first procedure Louis Reasoner wrote for calculating exponentiation modulo
{\tt m} is {\tt expmod-1}.  Smart as he is, he also wrote a procedure called
{\tt timed-exp}, which times how long it takes the computer to run a procedure
{\tt f} on arguments {\tt b e m}.  Use {\tt expmod-1} to calculate 10 to the
power of 511, modulo 7.  Then use {\tt timed-exp} to time this calculation.  Be
aware that {\sc Scheme} may need to pause during a calculation to do some
housekeeping.\footnote{This problem can be more pronounced on Athena with
Kerberos and X and Mailers and Zephyrs and \ldots which are all busy annoying
your computer while it's trying to do useful work.} This can distort the times
you measure.  To compensate for this you should make each measurement 2 or 3
times to see that you get close to the same time measurement each time.  To
avoid retyping the same expression many times, note that in the {\tt *Scheme*}
interactive buffer you can move your cursor to the end of a previously typed
expression then hit the {\tt EXECUTE} key to re-evaluate the expression.
Better, you may opt to define a procedure to do the repeated measurements for
you.  At any rate, prepare a chart of your results as follows:

\begin{center}
\begin{tabular}{||l|r|r|r||}
\hline
& $10^{511} \bmod 7$ & $10^{1023} \bmod 7$ & $10^{2047} \bmod 7$\\ \hline
{\tt expmod-1}& & & \\ \hline
{\tt expmod-2}& & & \\ \hline
{\tt expmod-3}& & & \\ \hline
\end{tabular}
\end{center}

Fill in the time it takes for {\tt expmod-1} to calculate the specified
numbers in the chart.

Louis is pleased with the result, but his supervisor Alyssa P.\ Hacker,
who has an M.S.\ degree from MIT, criticized this procedure, saying, 
``The order of growth of your procedure can be reduced considerably, using the
idea of successive squaring.''  She then suggested to Louis that he take a look 
at Section 1.2.4 in her favorite bed-time reading - {\bf Struggle and
Intimidation of Computer Programming}.  Louis quickly read the
section and wrote {\tt expmod-2}.  Test out the performance of his
{\tt expmod-2} and fill in the results in the prepared chart. 

Louis showed his {\tt expmod-2} to Ben Bitdiddle, who is a consultant from the
M.I.T Superficial Arrogance Lab.  Ben said, ``That's not bad, but you are not
taking any advantage of the fact that we are doing modular arithmetic!'' He
quickly wrote {\tt expmod-3} for Louis.  Test out {\tt expmod-3} and complete
the chart.

(a) Hand in a copy of your completed chart.

(b) What is the order of growth in time and space of {\tt expmod-1}, {\tt
expmod-2} and {\tt expmod-3}, measured in terms of the number of
multiplications?  Ignore the cost of evaluating {\tt remainder}.

(c) Explain why {\tt expmod-3} runs faster than both {\tt expmod-1} and
{\tt expmod-2}.  To do this, you will want to consider both the number
of multiplications and the cost of performing a multiplication.  (You
may recall from lecture that the cost of the latter operation
typically depends
on the size of the numbers being multiplied.)

%----------------------------------------------------------------------
\section{Exercise 2}

``Finally, my {\tt expmod} is working!'' said Louis with great relief.  He then
spent the rest of the day completing the key selection procedures.
Unfortunately, by the time he finished all this, the moonlight was filtering in
through his office window, and he fell asleep on his desk before he could test
out his work.  Please test out these procedures for Louis.  Simply run the {\tt
select-keys} procedure and generate some $n$ and $e$ numbers.  Notice that we
are not really using 100 digit prime numbers because of the limited capacity of
the Chipmunks.  Make a list of 4 pairs of $n$ and $e$ numbers as follows:

\begin{center}
\begin{tabular}{||l|r|r||} \hline
                & \multicolumn{1}{c|}{n} & \multicolumn{1}{c||}{e} \\ \hline
{\tt key-pair-1}&    $\hphantom{000000}$ &     $\hphantom{000000}$ \\ \hline
{\tt key-pair-2}&                        &                         \\ \hline
{\tt key-pair-3}&                        &                         \\ \hline
{\tt key-pair-4}&                        &                         \\ \hline
\end{tabular}
\end{center}

(a) Hand in a completed copy of the table.

(b) Is the {\tt gcd} procedure used in {\tt select-keys} an iterative
or recursive procedure? 

(c) Does the {\tt gcd} procedure give rise to an iterative or recursive
process? 

%----------------------------------------------------------------------
\section{Exercise 3}

\ldots Meanwhile, on the other side of the globe, two Israeli agents,
Artur O.\ Grossberg and Baruch Schtraktur, 
are burning the midnight oil to complete their implementation of the
code cracker for the CIA's new RSA system (their licensed-to-kill agent 6003
Sigmund Proceszeng alerted them to the CIA's recent activity).  Here's
what they have so far:

\beginlisp
(define crack-rsa
  (lambda (n e)
    (let ((start-time (runtime)))
      (let ((p (prime-component n)))
        (if (= p 1)
            (print "Sorry - cannot find prime component")
            (.................))))))
\null
(define prime-component
  (lambda (n)
    (define iter
      (lambda (guess)
        (...........)))
    (iter ...)))
\null
(define improve
  (lambda (guess)
    (-1+ guess)))
\endlisp

The dotted parts are not implemented yet.  In other words, these two characters
didn't come up with anything impressive, not to mention useful.  We ask you
in this exercise to complete these procedures, using the information we gave
you earlier.  Here is a summary of the highlights:

\begin{eqnarray*}
n = pq && \hbox{\rm ($p$ and $q$ are primes)}\\
m = (p - 1)(q - 1)&& \\
d = (1 - mr)/e && \hbox{\rm ($r$ is a negative integer)}
\end{eqnarray*}

{\tt Crack-rsa} takes the two public keys $n$ and $e$ as arguments,
and returns the decryption key $d$.  It calls {\tt prime-component} to
find out one of $n$'s prime components.  $q$ and $m$ can subsequently
be found.  Finding $d$ requires some systematic guessing, which
probably means that you will need some sort of recursive search
procedure.  Make use of the fact that $d$ must be an integer as the
stopping condition of the guessing.  The internal variable {\tt
start-time} is there for your convenience in putting a timing
mechanism into the procedure.  You may refer to the {\tt select-keys}
procedure as to how to do that.

{\tt Prime-component} takes a number $n$, and starts by using an integer
as the initial guess of one of the prime components of $n$.  The initial
guess, if not chosen appropriately, will cause the procedure to take
too long to run or produce number-out-of-bound error messages,  so think about
efficiency when you select the initial guess.  [Hint: the guess you
selected may not be an integer as a result of some calculations made
on $n$.  In that case, use the {\tt floor} {\sc Scheme} primitive to truncate
the numerals beyond the decimal point, so as to yield an integer.]
{\tt Prime-component} iterates until it hits a prime number $p$ such that
dividing $n$ by $p$ yields an integer.
{\tt Prime-component} should return this prime number
$p$. {\tt  Prime-component} calls the
procedure {\tt improve} to improve its guesses.  The given {\tt improve}
procedure, which decrements the guess by 1, is rather inefficient.  For
this part of the lab, however, you need not worry about changing the
{\tt improve} procedure since you will be asked to do so in the subsequent
exercises.  The fragmented code of {\tt crack-rsa} and {\tt prime-component}
are not given with the rest of the code, so you will need to type it in.
The next exercise will enable you to test out your procedures.  Be aware
that you can add your own internal defines if you feel it would be appropriate.

Hand in a listing of your procedures.

%----------------------------------------------------------------------
\section{Exercise 4}

Test out your {\tt crack-rsa} procedure by finding out the $d$ numbers
(or the secret key) of the following pairs of $n$ and $e$ numbers (or
the public keys).  Before you test out your procedures, make sure that
you have installed a timing mechanism in the {\tt crack-rsa} procedure
so that the {\tt crack-rsa} prints out the elapsed time when it
finds the answer.  Record the elapsed times and fill in the cracked keys in
the chart below in the ``time1'' column.

\begin{center}
\begin{tabular}{||r|r|r|r|r|r||} \hline
\multicolumn{1}{||c|}{n} &
\multicolumn{1}{  c|}{e} &
\multicolumn{1}{  c|}{d}          & time1 & time2 & time3 \\ \hline
 35369 & 697 & $\hphantom{00000}$ &       &       &       \\ \hline
 31789 & 859 &                    &       &       &       \\ \hline
 43931 &  17 &                    &       &       &       \\ \hline
146087 & 907 &                    &       &       &       \\ \hline
\end{tabular}
\end{center}

After you have found the secret keys to the corresponding public keys, choose
one set of keys to encrypt some random number and then decrypt the encrypted
number.  Verify that you get back the same number after the decryption.  Use
the {\tt encrypt} and {\tt decrypt} procedures provided in the code. (Caveat:
since we decrypt the encrypted message modulo $n$, you should restrict your
original message $s$ such that $s < n$; otherwise, the decryption will not
produce $s$ but rather $s \bmod n$.)

Hand in a {\sc Scheme} transcript (interactive session) of an
encryption-decryption execution.

%----------------------------------------------------------------------
\section{Exercise 5}

The Israelis sent out their 6002 agent Osczellas Gope to steal your
solution (assuming that if your solution has bugs, they could fix it)
and install it onto their system.  Baruch Schtraktur, who specialized
in algorithms at the Weizman Institute, immediately noticed that the
{\tt improve} procedure is very inefficient in that it doesn't skip over
even
numbers.  Modify it so that it skips over even numbers then run
the same timing tests as you did in the previous exercise and put the results
in the ``time2'' column.   You should find out how much more effective the new
{\tt improve} procedure is and report it directly to Alyssa P.\ Hacker.

Hand in a listing of your new {\tt improve} procedure.

%----------------------------------------------------------------------
\section{Exercise 6}

Ms.~Hacker has learned from you what the Israelis have done.  ``Now, for
something tough to crack \ldots'', said Ms.~Hacker, as she increased the
bounds of the variable $n$ in {\tt select-prime} to 40 and 10
respectively, i.e., $n$ is now chosen at random to lie between 10 and
40, then the formula $n^2+n+41$ is used to generate candidate
primes.  ``This way,
our RSA system will be generating bigger prime numbers, which should take
anyone considerably longer time to crack the decryption keys,'' she
explained with great confidence.

Modify the {\tt select-prime} procedure in your code according to
Alyssa,
and generate some new big keys, then use {\tt crack-rsa} to find the $d$
number
of these keys.  You'll realize that Alyssa is right (after all, she
{\it is} from MIT). Fill in the table below with the newly generated keys and
the results of cracking them, putting the times measured in the ``bigtime2''
column--- we won't go back and measure times for the original simple decrement
{\tt improve} so there is no ``bigtime1'' column.

\begin{center}
\begin{tabular}{||r|r|r|r|r||} \hline
\multicolumn{1}{||c|}{n} &
\multicolumn{1}{  c|}{e} &
\multicolumn{1}{  c|}{d} & bigtime2 & bigtime3 \\ \hline
\hphantom{00000000} & \hphantom{00000} & $\hphantom{0000000}$ & & \\ \hline
                    &                  &                      & & \\ \hline
                    &                  &                      & & \\ \hline
                    &                  &                      & & \\ \hline
\end{tabular}
\end{center}

The Israelis are very upset.  6002 and 6003 are so embarrassed that they are
both ready to hand in their ``drop-cards'' (resignation letters).  At this
point of despair, they learned that Louis Reasoner is going to France for
vacation.  They immediately set up agent 6034 R.~T.~Fishle to seduce Louis.
The Israelis hope that they can get some useful information out of Louis in the
process.  Ms.~Fishle was very successful (and lucky): Louis had recently
developed a serious case of sleep-talking syndrome (too many all-nighters).
While discussing hacking the rooftop of his hotel with Ms.~Fishle in the hotel
bar, Louis nodded off to sleep and not only gave away the new bounds in the
{\tt select-prime} procedure, he also gave away the equation $n^2+n+41$, which
he uses in his prime selection procedure.  Mr.~Grossberg exclaimed, ``Let's
improve our {\tt improve} procedure using this equation.'' You are asked to do
just that in this exercise.  Rewrite the {\tt improve} procedure such that it
systematically returns guesses which satisfy the formula $n^2+n+41$ for
$1<n<40$. (Notice that our lower bound here is $1$, not $10$, because we want
{\tt improve} to continue working on the keys that were generated when
{\tt select-prime} had a lower bound of $1$.)

(a) Hand in a listing of your new {\tt select-prime} and {\tt improve}
procedures and a {\sc Scheme} transcript of an encryption-decryption execution
on large keys generated by the {\tt select-keys} procedure.

(b) Do the timing test as you did in the previous exercise, putting the results
in the ``time3'' column, and hand in the completed chart.

(c) Now retime the cracking of the new larger keys you generated above.
Fill in the results in the ``bigtime3'' column and hand in a copy of the
completed chart.

(d) Briefly discuss the relative timings you measured for the two tables you
filled in.  Consider the orders of growth of the various {\tt improve}
strategies both considering multiples to be constant cost and considering them
to have cost proportional to the size of the numbers multiplied. You needn't
derive formal order-of-growth equations for this part: a concise
rationalization of your measurements will do. Be sure to address any unexpected
anomolies in your measurements. Specifically, does making the {\tt improve}
procedure increasingly more powerful always make {\tt crack-rsa} run faster? Do
any of these characteristics change as the keys generated become larger or do
they just become more pronounced? There is a wealth of experience to be gleaned
from these simple measurements so give it some careful consideration. Any silly
Hahvahd weenie can benchmark a program but it takes an MIT wizard to draw
reasonable conclusions from the measurements.

%----------------------------------------------------------------------
\section{Exercise 7}

Another member of the Isreali team, Abigail Schtrakson, glances at the code
written so far and says, ``Enough number crunching and algorithm massaging. Now let's
consider some `software engineering' issues.'' Abby, as her associates call
her, is a woman with keen foresight.  She recognizes that
in the code built so far, there are actually two very similar
search procedures.  One is used for finding $d$ given a value for $m$,
as part of {\tt crack-rsa}.  The other is for finding a prime
component.  She realizes that she should be able to capture this
common idea in a more general search procedure, with the following
characteristics:

\begin{itemize}

\item {\bf Input:} {\tt guess success-test next-guess returner}

\item {\bf Output:}  value returned by applying {\tt returner} to value of
{\tt guess} that passes {\tt success-test}

\item {\bf Contract:}  apply {\tt success-test} to a succession of guesses,
starting with {\tt guess} and generated by applying {\tt next-guess}
to previous guess in sequence until {\tt success-test} is satisfied.
Then return output as indicated.

\end{itemize}

Implement Abby's idea by defining a procedure called
{\tt general-search}.  Turn in a listing of your procedure. Show how
you would use this to define a new version of {\tt prime-component},
and how you would use it to replace whatever internal search method
you used in {\tt crack-rsa}.

%----------------------------------------------------------------------
\section{Exercise 8}

Abby also suggests another generalization, this time involving {\tt encrypt}
and {\tt decrypt}.  Rather than having a single such procedure that
can be used given any keys Abby would like to have a personalized
encrypter which takes a single argument  of a message and generates
an encrypted version of that message based on his keys.  To do this,
she needs a procedure, called {\tt make-encrypter}, that takes as
arguments two keys, {\tt e} and {\tt n}, and returns an encrypter that
can be applied to any message.

Write such a {\tt make-encrypter} and turn in a listing of your code.  Use it
to define an encrypter and decrypter then use them to encrypt then decrypt a
random message.  Turn in a transcript of your {\sc Scheme} session.

%----------------------------------------------------------------------
\section{Exercise 9}
Finally, the CIA is ready to use their system to send their first
message to Iraq.  They  manage to sell some cluster bombs to the Iraqis.
The first message consists of 3 numbers, which stand for the month, date and
year of the transaction.  The CIA encrypted them using Sad I Am Yousame's
public keys: $n = 146087, e = 907$.  The encrypted message is as follows:

\[90507 \qquad   117680  \qquad     131064.\]

Using the new {\tt crack-rsa} procedure you implemented in the previous
exercise, find out the decryption key $d$ of Yousame.  Then, find out
the date of this business transaction by decrypting the above message.

Hand in a transcript interaction-session of the cracking of keys and the
decryption.

%----------------------------------------------------------------------
\section{Exercise 10}

The Israelis tapped the wires and decoded the message also.  They also happen
to know that the next message will be sent from the Iraqis to the CIA,
informing them about the exact location of the transaction.   Since the 
encryption keys are public, Artur and Baruch decided to use the CIA's
public keys to encode a forgery message, which discloses a location where
the Israelis, instead of Yousame,  will be waiting to
receive the shipment of warheads.

It turns out that the Iraqis are not stupid when it comes to such things as
secret business deals.  Their agent Jokem Martyr, formerly a dean of computer
science at Boston University, developed a scheme which prevents forgery.
(Actually, this is a plagiarism: Rivest, Shamir and Adleman invented it.) It
works as follows:

\pagebreak

\begin{eqnarray*}
B:&& msg \Rightarrow_{encrypt} (msg)^{Bd}\Rightarrow_{encrypt}
(msg)^{Bd\cdot Ae} \Rightarrow_{decrypt}\\
&& \Rightarrow_{decrypt} (msg)^{Bd\cdot (Ae\cdot Ad)}
\Rightarrow_{decrypt} (msg)^{Bd\cdot Be} = msg :A
\end{eqnarray*}

Let $B$ stand for the Iraqis, and $A$ stand for the CIA.  $B$ first
encrypts its message by $B$'s secret key $Bd$.  Then they encrypt it
again using $A$'s public key $Ae$.  When $A$ receives this doubly encrypted
message, he can first decrypt it using his secret key $Ad$, and then
decrypt it once
again using $B$'s public key $Be$, which should yield the original message.
Since only $B$ knows $Bd$, nobody else could have encrypted a message
using $Bd$, which is only decodable by $Be$.  Therefore if the message
can be decrypted by $Be$, it must be encrypted using $Bd$, and thus must
be genuinely from $B$.

For this exercise, use the following information for keys:

\begin{center}
\begin{tabular}{lll}
$Bn$ = 146087  &  $Be$ = 907 &   $Bd$ = what you found out in the previous
exercise\\
$An$ = 612569  &  $Ae$ = 271 &   $Ad$ = (use {\tt crack-rsa} to find out)
\end{tabular}
\end{center}

24 hours later, the CIA received two messages:

\begin{center}
\begin{tabular}{rrrrr}
 First message: & 138811 & 541278 & 604581 & 298324 \\	% Forged Nicosia, Cyprus
Second message: & 478627 &  87832 & 393106 & 336014	% Real   Aqaba, Jordan
\end{tabular}
\end{center}

One of them is, of course, a forgery from the Israelis,  and has only
been encrypted using A's keys.  The other is the
genuine message from the Iraqis, and has been doubly encrypted as
described above.   The CIA is giving out a cash reward to
whoever can help them to identify the genuine message.  Since you know
everything by now, why don't you help them to find out which message 
is genuine?  [Hint: the message consists of 4 numbers: the degree and
orientation of the latitude, and the degree and orientation of the longitude
of the location of the transaction.  The numerals coding the directions
corresponds to English letters in the following manner:

\begin{center}
\begin{tabular}{rcl}
       numeral & English letter &  Meaning \\ \hline
         5     &        E       &    East  \\
        14     &        N       &    North \\
        19     &        S       &    South \\
        23     &        W       &    West
\end{tabular}
\end{center}

A number which is not in a sensible range after decryption is intuitively a
forgery.] 

Finally, find out where the Iraqis want the transaction to be and
where the Israelis want the transaction to be --- use a real world atlas
to do this.

The conclusion of the story depends, of course, on whether you get the
last exercise right or not.  Good Luck and Skill!

%%%%%%%%%%%%%%%%%%%%%%%%%%%%%%%%%%%%%%%%%%%%%%%%%%%%%%%%%%%%%%%%%%%%%%%%%%%%%%%

\end{document}




