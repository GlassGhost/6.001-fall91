\input ../6001mac				% On ALTDORF.AI.MIT.EDU

\def\bigfbox#1{%
  \vtop{\vbox{\hrule%
              \hbox{\vrule\kern3pt%
                    \vtop{\vbox{\kern3pt#1}\kern3pt}%
                    \kern3pt\vrule}}%
        \hrule}}

\begin{document}

\psetheader{Fall Semester, 1991}{Problem Set~9}

\medskip

\begin{flushleft}
Issued: November 14, 1991 \\
\smallskip
Tutorial preparation for: Weeks of November~18 and November~25 \\
Written assignment due: \fbox{Wednesday} November~27, 1991, in recitation \\
\end{flushleft}

\begin{flushleft}
Reading:
\begin{tightlist}

\item Chapter~5.

\end{tightlist}
\end{flushleft}

\paragraph{QUIZ ANNOUNCEMENT} Quiz~2 is on Wednesday, November~20, in Walker
Memorial Gymnasium (54--340) from 5--7 PM or 7--9 PM.  You may take the quiz
during either one of these two periods, but students taking the quiz during the
first period will not be permitted to leave the room until the end of the
period and students taking the quiz during the second period will not be
admitted late without a very good explanation.  The quiz is {\em closed book},
but you may bring with you one $8\frac{1}{2}\times 11$ inch sheet of paper on
which you have written notes.  You will not need your ``Don't Panic'' manual
for this quiz.  The quiz will cover all material from the beginning of the
semester through problem set~8 and all material presented in class through
recitation on November~8.  In addition, a quiz handout, with further details
about the quiz, has been distributed in lecture.  Bring it with you to the
quiz.  Questions will be strongly based on the problems that were assigned for
written homework and for tutorial.  The best way to study for the quiz is to
review the written homework and tutorial assignments and to familiarize
yourself with the quiz handout.  Because of the quiz, there will be no problem
set distributed next week.

This problem set covers the three major programs discussed in chapter~5: the
register-machine simulator, the explicit control evaluator, and
the compiler.  This is a great deal of code.  One important skill that
we hope you are beginning to master by now is the ability to modify a
large program (like the compiler) by identifying the relevant parts,
rather than attempting to master every detail of the entire program.
This $\$$kill is very highly valued in ``the real world''.

\bigskip

\bigfbox{\hbox{\vbox{\bf The lectures on this material will cover these
programs in the same order as the problem set, so you will have no trouble
starting the problem set early.  We {\bf STRONGLY RECOMMEND} that you begin
work on parts of the problem set as soon as we have covered material relevant
to that portion.  Note you need to do a small amount of lab work to prepare for
tutorial the week of November~25.  (See the beginning of part~3.) Also, bear in
mind that the hardest part of this problem set is in part~6, in which you will
extend the compiler.}}}

\newpage

The {\cf load-problem-set} command for problem set~9 will load the
following files into {\sc Scheme}.

{\cf PS9-SYNTAX.SCM }
These are {\sc Scheme} procedures that define the representation of
expressions and environments.  This is essentially the same syntax as
was used in the meta-circular evaluator, with a few additions required
by the explicit-control version.

{\cf PS9-REGSIM.SCM}
This is the register machine simulator discussed in section~5.1.5.  It
has been modified to include a monitored stack, as suggested in the
subsection ``Monitoring machine performance'' (p.~417), and to
complain about various errors in machine descriptions, as suggested in
exercise~5.9. In addition we added a simple tracing facility, a
counter for the number of machine operations executed, and a dynamic
display of the stack depth using Chipmunk graphics.  You needn't do
more than use the simulator as a ``black box,'' but we've included a
hardcopy printout of the code with this problem set in case you want
to play with it.

{\cf PS9-ECEVAL.SCM }
This is the explicit-control evaluator described in section~5.2 of the
book. All of the code has been collected here in the form of a
definition to be processed by the register machine simulator when the
file is loaded, so do not be surprised if the loading takes quite a
while. This version of the evaluator also has hooks in it to handle
code produced by the compiler, as described in section~5.3.6 of the
textbook.  The only other change from the code listed in section~5.2
is to define two additional registers, {\cf arg1} and {\cf arg2},
which you will use in part~6 of this problem set.

{\cf PS9-COMP.SCM }
This is the compiler, discussed in section~5.3.

{\cf PS9-OPEN.SCM } This will be loaded into {\sc Scheme} and
placed in a buffer, but you will not need it until you are ready to
modify the compiler in part~6 of this assignment.  You should study this code
carefully.  A listing is attached to this problem set.

\section{Part 1:  \fbox{Tutorial Preparation for the week of November~18}}

Register machines provide a means of handcrafting code specific to a
particular program.  In principle, this should lead to extremely
efficient code, since one can avoid the overhead that comes from the
need to handle general computations.

\problem{Tutorial Problem 1.1}  In tutorial, you should demonstrate that you
can implement three register machines.  Each machine should start with
its argument in register {\cf lst} and stop with the answer in
register {\cf val}.

The first machine implements an iterative procedure that counts the
number of elements in a list, according to the following method:

\beginlisp
(define (count-elements-iter lst)
  (define (loop lst val)
    (cond ((null? lst) val)
          (else (loop (cdr lst) (+ val 1)))))
  (loop lst 0))
\endlisp

The second machine implements a recursive algorithm that counts the
elements in a list:

\beginlisp
(define (count-elements-rec lst)
  (cond ((null? lst) 0)
        (else (+ 1 (count-elements-rec (cdr lst))))))
\endlisp

The third machine implements the recursive {\cf countatoms} procedure
(p.~98 of the text):

\beginlisp
(define (countatoms lst)
  (cond ((null? lst) 0)
        ((atom? lst) 1)
        (else (+ (countatoms (car lst))
                 (countatoms (cdr lst))))))
\endlisp

How do the running time and space of these machines grow as the size of
the input list increases? (What are various ways you might measure the
``size'' of the input list for {\cf countatoms}?)

Remember to sketch the data paths for each machine: giving just the controller
listing for a machine is only half the task.

As explained in the textbook, register machines are built with {\cf
define-machine} and can be run by typing {\cf (start <machine-name>)}.
See pages~394--405 for some example machine definitions, taking note
of the syntax for {\cf define-machine}.

When implementing register machines, you should observe the following
caveats:

\begin{itemize}
\item Every instruction is one of the following types:
{\cf assign}, {\cf branch}, {\cf goto}, {\cf save}, {\cf restore}.
(You should not need to use {\cf perform}.)

\item The only values that can be assigned to registers or tested in
branches are constants, fetches from registers, or primitive
operations applied to fetches from registers. No nested operations are
permitted, for example {\cf (assign val (null? (car (fetch arg1))))} is
not permitted since a call to {\cf car} was nested inside a call to {\cf
null?}.  The list of available
primitive operations is: {\cf car cdr cons eq? atom?  null? + - * / < > =}.

\end{itemize}

\section{Part 2:  Running register machines in the lab}

In lab, use the register machine simulator to test each of the three
machines you designed for tutorial. Record the statistics on the growth
in time (which we'll consider to be the number of machine operations
required) and space (which we'll consider to be the maximum stack depth)
required to run each machine.  In later parts of this problem set,
we'll ask you to compare these statistics with similar results for the
explicit-control evaluator and the compiler.

To use the simulator, load the code for problem set~9, then type in
your machine definition, for example,

\beginlisp
(define-machine count-elements-iter
  (registers \ldots)
  (controller
     \vdots ))
\endlisp

You'll find it convenient to define test procedures that load an input
into a machine, run the machine, print some statistics, and return the
result computed by the machine.  For example:

\beginlisp
(define (test-count-elements-iter input)
  (remote-assign count-elements-iter 'lst input)
  (start count-elements-iter)
  (initialize-stack count-elements-iter)
  (initialize-ops-counter count-elements-iter)
  (remote-fetch count-elements-iter 'val)
  )
\endlisp

{\cf Initialize-stack} and {\cf initialize-ops-counter} print
statistics and reset some of machine's local state variables used in
statistics gathering.

In addition to routines that gather statistics for stack usage and
total number of operations, there are some procedures to help you
debug your machines.  {\cf Remote-trace-reg-on} will show all
assignments to a specified machine register as they occur.  For
example, evaluating:

\beginlisp
(remote-trace-reg-on count-elements-iter 'lst)
\endlisp

before running your test procedure will show you all the changes to the
{\cf lst} register. To see even more stuff, try:

\beginlisp
(remote-trace-on count-elements-iter)
\endlisp

which will print each machine instruction as it is executed.  To get rid
of these traces, use {\cf remote-trace-reg-off} and {\cf
remote-trace-off}.

\problem{Problem 2.1}  Debug your two {\cf count-elements} machines and
record some statistics on the total number of machine operations and the maximum
stack depth required to compute the length of a list.  For each machine, derive
formulas for the number of operations, total stack pushes, and maximum
stack depth as a function of $n$, where $n$ is the length of the list
argument.

\problem{Problem 2.2} Debug your {\cf countatoms} machine.  Record
statistics on the performance of your machine when the inputs are
balanced binary trees.  These inputs can be generated by the procedure:

\beginlisp
(define (balanced-binary-tree depth)
  (if (= depth 1)
      (cons 'a 'a)
      (let ((branch (balanced-binary-tree (- depth 1))))
        (cons branch branch))))
\endlisp

Specifically, you should define this in {\sc Scheme} then use the
{\cf (remote-assign <machine> <register-name> (balanced-binary-tree <depth>))}
command to setup your inputs for the {\cf countatoms} machines.

What is the order of growth in time and space required:

(a) As a function of the depth of the tree?

(b) As a function of the total number of atoms in the tree?

(Optional: Derive explicit formulas for the number of operations, total stack
pushes, and maximum stack depth as a function the depth of the tree.)

\section{Part 3:  Running the explicit-control evaluator}

The explicit control evaluator is implemented as a register machine.
Thus, when you run it, you are running a {\sc Scheme} embedded in the
simulator embedded in {\sc Scheme}! (It will be slow.)

To start the machine, type {\cf (go)}. If you are using a Chipmunk,
then on the bottom of the screen you will see a graphical
representation of the depth of the stack. Note that the graph wraps
around back to the left side of the screen for long runs.  At the end of a run,
the system will print a few statistics. For example:

\beginlisp
==> (go)
(TOTAL-PUSHES = 0 MAXIMUM-DEPTH = 0)
(MACHINE-OPS = 8)
\null
EC-EVAL==> (define (f x) (+ x (* x x)))
F
(TOTAL-PUSHES = 3 MAXIMUM-DEPTH = 3)
(MACHINE-OPS = 37)
\null
EC-EVAL==> (f 10)
110
(TOTAL-PUSHES = 21 MAXIMUM-DEPTH = 8)
(MACHINE-OPS = 181)
\null
EC-EVAL==> quit
DONE
\endlisp

Notice that the explicit control evaluator has the prompt {\cf
EC-EVAL==>} to show you that you are not talking to the ordinary
{\sc Scheme} system. Try playing around with a few expressions. Use the
``alpha'' and ``graphics'' to help see the graph of the stack depth.
If the graph gets too cluttered, you can clear the screen with the
{\cf clear-graphics} procedure which has been installed as a primitive
operation. The other primitives are:

\beginlisp
    car cdr cons atom? eq? null? + - * / > < =
\endlisp

but it is easy to add more. See the procedure {\cf setup-environment}
in the {\cf ECEVAL} file.

There is no error handler for the evaluator, so the slightest mistake
will bounce you back out to {\sc Scheme}. To get back to the evaluator, just
hit a {\cf <cntl>G} and type {\cf (go)} again. Your old definitions
will still be defined.

You can turn on instruction tracing by doing

\beginlisp
(remote-trace-on explicit-control-evaluator)
\endlisp

Even with a very simple expression, this will generate an enormous
printout.  You can also set {\cf remote-trace-reg-on} for the {\cf
exp} register, or one of the other registers.

\medskip

\bigfbox{\hbox{\vbox{\problem{TO DO BEFORE TUTORIAL THE WEEK OF NOVEMBER~25} Turn on
remote trace for the explicit-control evaluator, then start the evaluator.
Define the following very simple function, and then give a very simple call to
the function.  Make a printout of the sequence of machine operations and bring
it with you to tutorial the week of November~25. See Part~4 below too.}}}

Here is the function and call that we would like you to use:

\beginlisp
EC-EVAL==> (define (f x) (+ x 2))
EC-EVAL==> (f 3)
\endlisp

Notice that in the midst of this tracing, you will be prompted for some input.
Because of all the verbiage generated by the tracing, you can easily get
confused here. To help avoid this confusion, consider the following partial
transcript of what you will see:

\beginlisp
==> (remote-trace-on explicit-control-evaluator)
()
\null
==> (go)
(PERFORM (INITIALIZE-STACK))
(TOTAL-PUSHES = 0 MAXIMUM-DEPTH = 0)
(PERFORM (INITIALIZE-OPS-COUNTER))
(MACHINE-OPS = 8)
(PERFORM (NEWLINE))
\null
(PERFORM (NEWLINE))
\null
(PERFORM (PRINC "EC-EVAL==> "))EC-EVAL==>                ;; Prompted here
(ASSIGN EXP (READ-FROM-KEYBOARD))(define (f x) (+ x 2))  ;; Waited for input here
(BRANCH (EXIT-ON? (FETCH EXP)) EC-EVAL-EXIT)
(ASSIGN ENV THE-GLOBAL-ENVIRONMENT)
(ASSIGN CONTINUE PRINT-RESULT)
(GOTO EVAL-DISPATCH)
     \vdots
\endlisp

\medskip

\bigfbox{\hbox{\vbox{Be sure to turn off tracing before continuing with this
part of the problem set.}}}

\medskip

\problem{Problem 3.1} Start the explicit-control evaluator and,
typing at the evaluator, define the procedures {\cf
count-elements-iter,} {\cf count-elements-rec,} and {\cf countatoms}
listed in part~1.  Record statistics like the ones you did in part~2, to
determine the growth in time and space for these procedures.
Determine explicit formulas for the number of stack pushes, maximum
stack depth, and total machine instructions required to count the
elements in a list of length $n$ using the two {\cf count-elements}
procedures.

\problem{Problem 3.2} Record some statistics on the work required by
{\cf countatoms} with a few simple inputs, and compare this with the
work required by your {\cf countatoms} register machine from part~2.
(Optional: Derive formulas for the work required by {\cf countatoms}
on balanced binary trees, as in problem~2.2.  Be careful not to
include the work required to generate the tree---just the work to
count the atoms.)


\section{Part 4:  \fbox{Tutorial Preparation for the week of November~25}}

\problem{Problem 4.1} Take the list of operations you generated
(see beginning of part~3 above) and explain which operations would be
omitted if these expressions were compiled rather than interpreted.

\problem{Problem 4.2} The following is the output of the compiler
for a simple function definition.  Give the {\sc Scheme} source expression
that was compiled to produce this output:

\beginlisp
((assign val (make-compiled-procedure entry4 (fetch env)))
 (goto after-lambda3)
\null
 entry4
 (assign env (compiled-procedure-env (fetch fun)))
 (assign env (extend-binding-environment '(x) (fetch argl) (fetch env)))
 (save env)
 (assign fun (lookup-variable-value '= (fetch env)))
 (assign val (lookup-variable-value 'x (fetch env)))
 (assign argl (cons (fetch val) '()))
 (assign val '0)
 (assign argl (cons (fetch val) (fetch argl)))
 (assign continue after-call6)
 (save continue)
 (goto apply-dispatch)
\endlisp
% Allow page break
\beginlisp
\null
 after-call6
 (restore env)
 (branch (true? (fetch val)) true-branch5)
 (assign fun (lookup-variable-value '+ (fetch env)))
 (save fun)
 (assign val (lookup-variable-value 'x (fetch env)))
 (assign argl (cons (fetch val) '()))
 (save argl)
 (assign fun (lookup-variable-value 'foo (fetch env)))
 (save fun)
 (assign fun (lookup-variable-value '- (fetch env)))
 (assign val (lookup-variable-value 'x (fetch env)))
 (assign argl (cons (fetch val) '()))
 (assign val '1)
 (assign argl (cons (fetch val) (fetch argl)))
 (assign continue after-call8)
 (save continue)
 (goto apply-dispatch)
\endlisp
% Allow page break
\beginlisp
\null
 after-call8
 (assign argl (cons (fetch val) '()))
 (restore fun)
 (assign continue after-call7)
 (save continue)
 (goto apply-dispatch)
\null
 after-call7
 (restore argl)
 (assign argl (cons (fetch val) (fetch argl)))
 (restore fun)
 (goto apply-dispatch)
\null
 true-branch5
 (assign val '-1)
 (restore continue)
 (goto (fetch continue))
\null
 after-lambda3
 (perform (define-variable! 'foo (fetch val) (fetch env)))
 (assign val 'foo)
 (restore continue)
 (goto (fetch continue)))
\endlisp

\section{Part 5: Using the compiler}

The compiler translates {\sc Scheme} expressions into register-machine code
that can be run as part of the explicit-control evaluator machine
(check out the book!). You can look at the output of the compiler by
using the {\sc Scheme} {\cf compile} procedure. For example:

\beginlisp
==> (pp (compile '(* x y)))
((ASSIGN FUN (LOOKUP-VARIABLE-VALUE '* (FETCH ENV)))
 (ASSIGN VAL (LOOKUP-VARIABLE-VALUE 'X (FETCH ENV)))
 (ASSIGN ARGL (CONS (FETCH VAL) '()))
 (ASSIGN VAL (LOOKUP-VARIABLE-VALUE 'Y (FETCH ENV)))
 (ASSIGN ARGL (CONS (FETCH VAL) (FETCH ARGL)))
 (GOTO APPLY-DISPATCH))
\endlisp

To actually run the code, use the procedure {\cf compile-and-go} which
compiles an expression and runs the result via the explicit-control
evaluator:

\beginlisp
==> (compile-and-go '(* 5 (+ 4 6)))
(TOTAL-PUSHES = 0 MAXIMUM-DEPTH = 0)
50
(TOTAL-PUSHES = 4 MAXIMUM-DEPTH = 4)
(MACHINE-OPS = 40)
\null
EC-EVAL==> punt
DONE
\endlisp

You can compile definitions with {\cf compile-and-go} which will run
the {\cf define} and make the compiled procedure available from the
evaluator. (Since {\cf compile} and {\cf compile-and-go} are
top-level to {\sc Scheme}, you can {\cf zap} explicit calls to them from
the editor instead of typing them by hand.).

\beginlisp
==> (compile-and-go '(define (f x) (+ x 5)))
(TOTAL-PUSHES = 0 MAXIMUM-DEPTH = 0)
F
(TOTAL-PUSHES = 1 MAXIMUM-DEPTH = 1)
(MACHINE-OPS = 21)
\null
EC-EVAL==> (f 10)
15
(TOTAL-PUSHES = 5 MAXIMUM-DEPTH = 3)
(MACHINE-OPS = 68)
\null
EC-EVAL==> fini
DONE
\endlisp

\problem{Problem 5.1} Compile your three {\cf count} procedures and
do an analysis similar to the ones you did in parts~2 and~3 to find
the number of machine operations and the amount of space required.
Use a simple fragment from any of the procedures to illustrate why the
compiled code runs faster and takes less space than the interpreted
code.  Does compiling the code ever change the {\em order of growth}
of the time or space required?

\problem{Problem 5.2} For each of the two {\cf count-elements}
procedures, (and optionally for {\cf countatoms}) prepare a table like
the following, drawing on the statistics you took in parts~2 and~3:

\beginlisp
                            COUNT-ELEMENTS-ITER on a list of length n
\null
                  number of ops      number stack pushes      max stack depth
--------------------------------------------------------------------------------
HAND-CODED     |                |                         |
---------------|----------------|-------------------------|---------------------
INTERPRETED    |                |                         |
---------------|----------------|-------------------------|---------------------
COMPILED       |                |                         |
--------------------------------------------------------------------------------
                       limit as n becomes large, of the ratio
--------------------------------------------------------------------------------
INTERPRETED/   |                |                         |
HAND-CODED     |                |                         |
---------------|----------------|-------------------------|---------------------
COMPILED/      |                |                         |
INTERPRETED    |                |                         |
---------------|----------------|-------------------------|---------------------
COMPILED/      |                |                         |
HAND-CODED     |                |                         |
--------------------------------------------------------------------------------
\endlisp

Briefly (in a short paragraph or two) summarize what insights you have gleaned
from these statistics.  This is probably one of the most important things you
will learn in this course from the point of few of a computer programmer.  Give
it some careful consideration.

\newpage

\section{Part 6: Open-coded primitive operations}

As you've seen in part~5, the book's compiler produces a big
improvement over interpreted code, but it has a long way to go to
match the performance of hand-coded programs.

The compiler worries mostly about avoiding unnecessary stack
operations.  There are lots of other ways in which compilers can
improve performance.  One of these is in making variable lookup more
efficient.\footnote{Note that the performance difference between
compiled and handcrafted code is actually worse than is shown by the
stack and operation counts. This is because {\cf lookup-variable-value}, which
in a real register machine would require many instructions, is counted
in our model as a single operation.} You can read about this in section~5.3.7
of the text.

In this part of the problem set, you'll consider another important
optimization---{\em open-coded primitives}.  This idea is described in
exercise~5.33 (p.~479).  You should read that exercise, although the
detailed implementation of open coding we shall use is different from
the one described in that exercise.

The file {\cf ps9-open} is placed into an editor buffer and
loaded into {\sc Scheme} when you do Load Problem Set~9
(and a listing has been included with this problem set).
Comments in the file describe the open coding method, and the file contains the
complete implementation for open coding calls to primitives with one
argument.  You will need to define the procedure {\cf
open-code-binary-primitive-application} to handle primitives with two
arguments.  A template for its definition has placed in your {\cf
ps9-answer} buffer. In this open-coding implementation, we'll worry
only about primitive applications where there are either one or two operands.

The explicit-control evaluator machine that was loaded with this
problem set includes two additional registers not found in the
definition in the book.  These are called {\cf arg1} and {\cf arg2}
and you will use them in compiling open-coded primitives, as described
in {\cf ps9-open}.

The open-coding implementation includes a flag called {\cf
open-code?}, initially set to {\cf false}, that controls whether or
not to use open coding.  After you get your code running, you can turn
this on and off (using {\cf (open-code!)} and {\cf (close-code!)}) to compare the
results of open coding and
closed coding. (``Closed-coded'' means that the call is not open coded.)

\problem{Problem 6.1} Complete the definition of 
{\cf open-code-binary-primitive-application} in your {\cf ps9-answer} buffer.
Now zap your definition into {\sc Scheme}. Please think carefully about
your implementation.  Compilers are one area where it is {\em much} easier
to reason things out carefully beforehand.  You really don't want to
slog through buggy compiler output in an attempt to figure out what you
did wrong.  Test your open-coding method by setting {\cf open-code?}
to true then trying {\cf (compile-and-go '(+ (* 3 4) (* 5 6)))}.

\problem{Problem 6.2} Evaluate {\cf (compile '(+ (* 3 4) (* 5 6)))}
both with and without open coding, and compare the results to
convince yourself that your open coding works as desired.  Turn in
listings of both results, marked to indicate where the differences are
in the generated code.

\problem{Problem 6.3} Compile {\cf count-elements-rec} both with and without
open coding, and compare the number of operations required for some simple
inputs.  Extend the table you made in problem~5.2 for {\cf count-elements-rec}
to compare the results of open coding with ordinary compilation.

\problem{Problem 6.4} Compile {\cf countatoms} with and without open
coding, and compare the behavior for a few simple inputs.

\problem{Warning/Hint} If your open coding produced the right answer
for the arithmetic expression in problem~6.1, but the wrong answer in~6.2
or~6.3, then you probably did not correctly preserve the required registers.

\end{document}


